\documentclass{article}
\usepackage[utf8]{inputenc}
\usepackage{amsmath}
\usepackage{amsfonts}
\usepackage{graphicx}

\title{Exemplo de Mistura Gaussiana para Classificação por Altura}
\author{}
\date{}

\begin{document}

\maketitle

\section*{Enunciado da Questão}
Considere uma população de 5 pessoas com as seguintes alturas (em cm): 165, 182, 168, 178 e 171. Você deve usar um modelo de mistura gaussiana (GMM) para separar essa população em dois grupos distintos com base na altura. Interprete os grupos resultantes como "adolescentes" e "adultos".

\section{Dados do Problema}
Considere uma população de 5 pessoas com as seguintes alturas (em cm):
\begin{itemize}
    \item Pessoa 1: 165
    \item Pessoa 2: 182
    \item Pessoa 3: 168
    \item Pessoa 4: 178
    \item Pessoa 5: 171
\end{itemize}

Queremos usar um modelo de mistura gaussiana (GMM) para separar essa população em 2 grupos distintos baseados na altura, interpretando-os como "adolescentes" e "adultos".

\section{Passo 1: Inicialização dos Parâmetros}
Inicializamos os seguintes parâmetros para os grupos de adolescentes e adultos:

\subsection{Parâmetros para Adolescentes (Grupo 1)}
\begin{itemize}
    \item Peso de mistura: $\pi_1 = 0.5$
    \item Média: $\mu_1 = 167$ cm
    \item Variância: $\sigma_1^2 = 25$ cm$^2$
\end{itemize}

\subsection{Parâmetros para Adultos (Grupo 2)}
\begin{itemize}
    \item Peso de mistura: $\pi_2 = 0.5$
    \item Média: $\mu_2 = 180$ cm
    \item Variância: $\sigma_2^2 = 25$ cm$^2$
\end{itemize}

\section{Passo 2: Algoritmo EM (Expectation-Maximization)}

\subsection{Iteração 1}

\subsubsection{Passo E (Expectation): Cálculo das Responsabilidades}
Para cada altura $x_i$, calculamos a probabilidade de pertencer a cada grupo:

\begin{equation}
p(x_i|k) = \frac{1}{\sqrt{2\pi\sigma_k^2}} \exp\left(-\frac{(x_i - \mu_k)^2}{2\sigma_k^2}\right)
\end{equation}

\begin{equation}
\gamma(i,k) = \frac{\pi_k \cdot p(x_i|k)}{\pi_1 \cdot p(x_i|1) + \pi_2 \cdot p(x_i|2)}
\end{equation}

\paragraph{Cálculos para a Pessoa 1 ($x_1 = 165$):}
\begin{align*}
p(x_1|1) &= \frac{1}{\sqrt{50\pi}} \exp\left(-\frac{(165-167)^2}{50}\right) = 0.0736 \\
p(x_1|2) &= \frac{1}{\sqrt{50\pi}} \exp\left(-\frac{(165-180)^2}{50}\right) = 0.0009 \\
\gamma(1,1) &= \frac{0.5 \cdot 0.0736}{0.5 \cdot 0.0736 + 0.5 \cdot 0.0009} = 0.9877 \\
\gamma(1,2) &= \frac{0.5 \cdot 0.0009}{0.5 \cdot 0.0736 + 0.5 \cdot 0.0009} = 0.0123
\end{align*}

\paragraph{Cálculos para a Pessoa 2 ($x_2 = 182$):}
\begin{align*}
p(x_2|1) &= \frac{1}{\sqrt{50\pi}} \exp\left(-\frac{(182-167)^2}{50}\right) = 0.0009 \\
p(x_2|2) &= \frac{1}{\sqrt{50\pi}} \exp\left(-\frac{(182-180)^2}{50}\right) = 0.0736 \\
\gamma(2,1) &= \frac{0.5 \cdot 0.0009}{0.5 \cdot 0.0009 + 0.5 \cdot 0.0736} = 0.0123 \\
\gamma(2,2) &= \frac{0.5 \cdot 0.0736}{0.5 \cdot 0.0009 + 0.5 \cdot 0.0736} = 0.9877
\end{align*}

\paragraph{Cálculos para a Pessoa 3 ($x_3 = 168$):}
\begin{align*}
p(x_3|1) &= \frac{1}{\sqrt{50\pi}} \exp\left(-\frac{(168-167)^2}{50}\right) = 0.0782 \\
p(x_3|2) &= \frac{1}{\sqrt{50\pi}} \exp\left(-\frac{(168-180)^2}{50}\right) = 0.0045 \\
\gamma(3,1) &= \frac{0.5 \cdot 0.0782}{0.5 \cdot 0.0782 + 0.5 \cdot 0.0045} = 0.9450 \\
\gamma(3,2) &= \frac{0.5 \cdot 0.0045}{0.5 \cdot 0.0782 + 0.5 \cdot 0.0045} = 0.0550
\end{align*}

\paragraph{Cálculos para a Pessoa 4 ($x_4 = 178$):}
\begin{align*}
p(x_4|1) &= \frac{1}{\sqrt{50\pi}} \exp\left(-\frac{(178-167)^2}{50}\right) = 0.0071 \\
p(x_4|2) &= \frac{1}{\sqrt{50\pi}} \exp\left(-\frac{(178-180)^2}{50}\right) = 0.0736 \\
\gamma(4,1) &= \frac{0.5 \cdot 0.0071}{0.5 \cdot 0.0071 + 0.5 \cdot 0.0736} = 0.0876 \\
\gamma(4,2) &= \frac{0.5 \cdot 0.0736}{0.5 \cdot 0.0071 + 0.5 \cdot 0.0736} = 0.9124
\end{align*}

\paragraph{Cálculos para a Pessoa 5 ($x_5 = 171$):}
\begin{align*}
p(x_5|1) &= \frac{1}{\sqrt{50\pi}} \exp\left(-\frac{(171-167)^2}{50}\right) = 0.0579 \\
p(x_5|2) &= \frac{1}{\sqrt{50\pi}} \exp\left(-\frac{(171-180)^2}{50}\right) = 0.0158 \\
\gamma(5,1) &= \frac{0.5 \cdot 0.0579}{0.5 \cdot 0.0579 + 0.5 \cdot 0.0158} = 0.7857 \\
\gamma(5,2) &= \frac{0.5 \cdot 0.0158}{0.5 \cdot 0.0579 + 0.5 \cdot 0.0158} = 0.2143
\end{align*}

\section{Resultados após a Primeira Iteração}

Cada valor de $\gamma(i,k)$ representa a probabilidade de a altura $x_i$ pertencer ao grupo $k$.

\section{Passo 3: Classificação Final}
Para classificar cada pessoa, comparamos as responsabilidades:

\begin{itemize}
    \item Pessoa 1 (165 cm): Grupo 1 ($\gamma(1,1) = 0.9877 > \gamma(1,2) = 0.0123$)
    \item Pessoa 2 (182 cm): Grupo 2 ($\gamma(2,2) = 0.9877 > \gamma(2,1) = 0.0123$)
    \item Pessoa 3 (168 cm): Grupo 1 ($\gamma(3,1) = 0.9450 > \gamma(3,2) = 0.0550$)
    \item Pessoa 4 (178 cm): Grupo 2 ($\gamma(4,2) = 0.9124 > \gamma(4,1) = 0.0876$)
    \item Pessoa 5 (171 cm): Grupo 1 ($\gamma(5,1) = 0.7857 > \gamma(5,2) = 0.2143$)
\end{itemize}

\section{Conclusão}
A separação em dois grupos ficou:
\begin{itemize}
    \item Grupo 1 (adolescentes): Pessoas 1, 3 e 5 com alturas 165 cm, 168 cm e 171 cm
    \item Grupo 2 (adultos): Pessoas 2 e 4 com alturas 182 cm e 178 cm
\end{itemize}

Em um caso real, continuaríamos as iterações até a convergência, mas já nesta primeira iteração conseguimos observar uma separação clara dos grupos.

\end{document}
